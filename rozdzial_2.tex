\newgeometry{left=1cm,top=1.5cm}
\chapter{Podstawy prawne}
	\section{Dyrektywa INSPIRE}
Najważniejszym dokumentem definiującym podstawy prawne dla działań Open Data w Polsce jest Dyrektywa INSPIRE\footnote{DYREKTYWA 2007/2/WE PARLAMENTU EUROPEJSKIEGO I RADY z dnia 14 marca 2007 r. ustanawiająca infrastrukturę informacji przestrzennej we Wspólnocie Europejskiej (INSPIRE)}. INfrastructure for SPatial InfoRmation in Europe (Infrastruktura Informacji Przestrzennej w Europie) to ratyfikowane w 2007, obowiązujące we wszystkich państwach członkowskich wspólnoty od conajmniej 2009 roku ramy, które określają jakie zbiory danych, metadanych, oraz usług powinny być dostępne dla wszystkich podmiotów. 
\\Główne założenia INSPIRE:
\begin{itemize}
\item Dane powinny być pozyskiwane tylko jeden raz oraz przechowywane i zarządzane w sposób najbardziej poprawny i efektywny przez odpowiednie instytucje i służby.
\item Powinna być zapewniona ciągłość przestrzenna danych tak, aby było możliwe pozyskanie różnych zasobów, z różnych źródeł oraz aby możliwe było ich udostępnianie wielu użytkownikom i do różnorodnych zastosowań.
\item Dane przestrzenne powinny być przechowywane na odpowiednim (jednym) poziomie administracji publicznej i udostępniane podmiotom na wszystkich pozostałych poziomach.
\item Dane przestrzenne niezbędne do odpowiedniego zarządzania przestrzenią na wszystkich poziomach administracji publicznej powinny być powszechnie dostępne (tj. bez warunków ograniczających i/lub utrudniających ich swobodne wykorzystanie).
\item Powinien być zapewniony dostęp do informacji o tym, jakie dane przestrzenne są dostępne i na jakich
warunkach, a także informacja umożliwiająca użytkownikowi ocenę przydatności tych danych do swoich celów.
\end{itemize}

\section{Ustawa o IIP}
Kolejnym dokumentem definiującym zasady tworzenia danych przestrzennych, realizacji usług a także organy administracji publicznej zoobowiązane do przygotowania elementów infrastruktury INSPIRE jest Ustawa o infrastrukturze informacji przestrzennej\footnote{Ustawa z dnia 4 marca 2010 r. o infrastrukturze informacji przestrzennej (Dz.U. 2017.1382)}. Większość treści ustawy jest bezpośrednią transpozycją rozwiązań dyrektywy. Dla naszych rozważań istotne są dwa artykuły:

\begin{prawne}
Art. 5. 1. Tworzenie, aktualizacja i udostępnianie zbiorów metadanych infrastruktury, zwanych dalej „metadanymi”, jest zadaniem organów administracji, odpowiedzialnych w zakresie swojej właściwości za prowadzenie rejestrów publicznych zawierających zbiory związane z wymienionymi w załączniku do ustawy tematami danych przestrzennych, oraz osób trzecich, których zbiory włączane są do infrastruktury.
2. Metadane obejmują informacje dotyczące w szczególności:
1) zgodności zbiorów z obowiązującymi przepisami, dotyczącymi tematów danych
przestrzennych określonych w załączniku do ustawy;
2) warunków uzyskania dostępu do zbiorów i ich wykorzystania, usług danych przestrzennych oraz wysokości opłat, jeżeli są pobierane;
3) jakości i ważności zbiorów w rozumieniu ust. 2 w części A załącznika do rozporządzenia Komisji (WE) nr 1205/2008 z dnia 3 grudnia 2008 r. w sprawie wykonania dyrektywy nr 2007/2/WE Parlamentu Europejskiego i Rady
w zakresie metadanych (Dz. Urz. UE L 326 z 04.12.2008, str. 12);
4) organów administracji odpowiedzialnych za tworzenie, aktualizację i udostępnianie zbiorów oraz usług danych przestrzennych;
5) ograniczeń powszechnego dostępu do zbiorów i usług danych przestrzennych oraz przyczyn tych ograniczeń.	
\end{prawne}
\begin{prawne}
Art. 12. 1. Dostęp do usług, o których mowa w art. 9 ust. 1 pkt 1 i 2, jest powszechny i nieodpłatny
\end{prawne}
\section{Prawo geodezyjne i kartograficzne}
Prawo geodezyjne i kartograficzne\footnote{Ustawa z dnia 17 maja 1989r. Prawo Geodezyjne i Kartograficzne(tekst jednolity: Dz.U. 2017 poz. 1566)} jest ustawą definiującą sposób organizacji służby geodezyjnej i kartograficznej oraz państwowego zasobu geodezyjnego i kartograficznego. Szczególnie istotny dla nas jest art. 40a który wskazuje literalnie "otwarte dane" należące do PZGiK.

\begin{prawne}
	Art. 40a. 1. Organy prowadzące państwowy zasób geodezyjny i kartograficzny udostępniają materiały zasobu odpłatnie.
	\\2. Nie pobiera się opłaty za:
	\\1) udostępnianie zbiorów danych:
\\	a) państwowego rejestru granic i powierzchni jednostek podziałów terytorialnych kraju,
\\	b) państwowego rejestru nazw geograficznych,
\\	c) zawartych w bazie danych obiektów ogólnogeograficznych,
\\	d) dotyczących numerycznego modelu terenu o interwale siatki co najmniej 100 m;
\end{prawne}

\section{Dostęp do informacji publicznej}

Ustawa z dnia 6 września 2001 r. o dostępie do informacji publicznej, (Dz.U. 2016 poz. 1764)

\begin{prawne}
Art. 4. 1. Obowiązane do udostępniania informacji publicznej są władze publiczne oraz inne podmioty wykonujące zadania publiczne, w szczególności:
\\1) organy władzy publicznej;
\\2) organy samorządów gospodarczych i zawodowych;
\\3) podmioty reprezentujące zgodnie z odrębnymi przepisami Skarb Państwa;
\\4) podmioty reprezentujące państwowe osoby prawne albo osoby prawne samorządu terytorialnego oraz podmioty reprezentujące inne państwowe jednostki organizacyjne albo jednostki organizacyjne samorządu terytorialnego;
\\5) podmioty reprezentujące inne osoby lub jednostki organizacyjne, które wykonują zadania publiczne lub dysponują majątkiem publicznym, oraz osoby prawne, w których Skarb Państwa, jednostki samorządu terytorialnego lub samorządu gospodarczego albo zawodowego mają pozycję dominującą w rozumieniu przepisów o ochronie konkurencji i konsumentów.
\\2. Obowiązane do udostępnienia informacji publicznej są organizacje związkowe i pracodawców, reprezentatywne w rozumieniu ustawy z dnia 24 lipca 2015 r. o Radzie Dialogu Społecznego i innych instytucjach dialogu społecznego (Dz. U. poz. 1240), oraz partie polityczne.
\\3. Obowiązane do udostępniania informacji publicznej są podmioty, o których mowa w ust. 1 i 2, będące w posiadaniu takich informacji.	 
\end{prawne}

\begin{prawne}

Art. 6. 1. Udostępnieniu podlega informacja publiczna, w szczególności o:

1) polityce wewnętrznej i zagranicznej, w tym o:
a) zamierzeniach działań władzy ustawodawczej oraz wykonawczej,
(...)
c) programach w zakresie realizacji zadań publicznych, sposobie ich realizacji, wykonywaniu i skutkach realizacji tych zadań;
(...)
4) danych publicznych, w tym:
a) treść i postać dokumentów urzędowych, w szczególności:
– treść aktów administracyjnych i innych rozstrzygnięć,
(...)
b) stanowiska w sprawach publicznych zajęte przez organy władzy publicznej i przez funkcjonariuszy publicznych w rozumieniu przepisów Kodeksu karnego,
c) treść innych wystąpień i ocen dokonywanych przez organy władzy publicznej,
d) informacja o stanie państwa, samorządów i ich jednostek organizacyjnych;
5) majątku publicznym, w tym o:
a) majątku Skarbu Państwa i państwowych osób prawnych,
b) innych prawach majątkowych przysługujących państwu i jego długach,
c) majątku jednostek samorządu terytorialnego oraz samorządów zawodowych i gospodarczych oraz majątku osób prawnych samorządu terytorialnego, a także kas chorych d) majątku podmiotów, o których mowa w art. 4 ust. 1 pkt 5, pochodzącym z zadysponowania majątkiem, o którym mowa w lit. a–c, oraz pożytkach z tego majątku i jego obciążeniach,
(...)

Art. 7. 1. Udostępnianie informacji publicznych następuje w drodze:
1) ogłaszania informacji publicznych, w tym dokumentów urzędowych, w Biuletynie Informacji Publicznej, o którym mowa w art. 8;
2) udostępniania, o którym mowa w art. 10 i 11;
3) wstępu na posiedzenia organów, o których mowa w art. 3 ust. 1 pkt 3, i udostępniania materiałów, w tym audiowizualnych i teleinformatycznych, dokumentujących te posiedzenia;
4) udostępniania w centralnym repozytorium.
2. Dostęp do informacji publicznej jest bezpłatny, z zastrzeżeniem art. 15.

Art. 13. 1. Udostępnianie informacji publicznej na wniosek następuje bez zbędnej zwłoki, nie później jednak niż w terminie 14 dni od dnia złożenia wniosku, z zastrzeżeniem ust. 2 i art. 15 ust. 2.
\end{prawne}
\begin{prawne}
Art. 15. 1. Jeżeli w wyniku udostępnienia informacji publicznej na wniosek,
o którym mowa w art. 10 ust. 1, podmiot obowiązany do udostępnienia ma ponieść
dodatkowe koszty związane ze wskazanym we wniosku sposobem udostępnienia lub
koniecznością przekształcenia informacji w formę wskazaną we wniosku, podmiot ten
może pobrać od wnioskodawcy opłatę w wysokości odpowiadającej tym kosztom.
2. Podmiot, o którym mowa w ust. 1, w terminie 14 dni od dnia złożenia
wniosku, powiadomi wnioskodawcę o wysokości opłaty. Udostępnienie informacji
zgodnie z wnioskiem następuje po upływie 14 dni od dnia powiadomienia
wnioskodawcy, chyba że wnioskodawca dokona w tym terminie zmiany wniosku
w zakresie sposobu lub formy udostępnienia informacji albo wycofa wniosek.
\end{prawne}
Ustawa z dnia 25 lutego 2016 r. o ponownym wykorzystywaniu informacji sektora publicznego
, Dz.U. 2016 poz. 352
\begin{prawne}
Art. 13. 1. Informacje sektora publicznego udostępnia się lub przekazuje w celu
ich ponownego wykorzystywania bezwarunkowo.\end{prawne}
\begin{prawne}
Art. 14. 1. Warunki ponownego wykorzystywania mogą dotyczyć:
1) obowiązku poinformowania o źródle, czasie wytworzenia i pozyskania
informacji od podmiotu zobowiązanego;
2) obowiązku poinformowania o przetworzeniu informacji ponownie
wykorzystywanej;
3) zakresu odpowiedzialności podmiotu zobowiązanego za udostępniane lub
przekazywane informacje.\end{prawne}
\begin{prawne}
Art. 16. Informacje sektora publicznego udostępnia się lub przekazuje w celu
ich ponownego wykorzystywania bezpłatnie.\end{prawne}
\begin{prawne}
Art. 17. 1. Podmiot zobowiązany może nałożyć opłatę za ponowne
wykorzystywanie, jeżeli przygotowanie lub przekazanie informacji w sposób lub
w formie wskazanych we wniosku o ponowne wykorzystywanie wymaga
poniesienia dodatkowych kosztów.\end{prawne}
