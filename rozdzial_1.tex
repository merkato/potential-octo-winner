\newgeometry{left=1cm,top=1.5cm}
\chapter{Co to są otwarte dane?}
\section{Zdefiniujmy pojęcie}
Nasze rozważania na temat otwartych danych zacznijmy od definicji pojęcia "otwarte" i "wolne". Podstawą niech będzie "otwarta definicja" - publikacja fundacji "Open Knowledge Foundation"\footnote{http://opendefinition.org/od/2.0/pl/}
\begin{mdframed}[backgroundcolor=code-gray]
Otwarte oznacza że każdy ma prawo dostępu, wykorzystania, zmian, i współdzielenia w dowolnym celu (pod warunkiem, co najwyżej, informowania o pochodzeniu danych i otwartości).
\end{mdframed}
Definicja w takiej formie powstała w roku 2004. Obecnie obowiązuje wersja 2.1 tego dokumentu. 

\section{Historia otwartości}
\section{Open Source}

Źródłem dla idei Open Data jest ruch Free Software, a następnie Open Source. ]
\section{Open Data}
	Podstawowe ustawienia	 


